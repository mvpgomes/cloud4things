\section{Evaluation Methodology}
\label{sec:evaluation}
The evaluation of the solution can be performed according two perspectives, one consists in evaluate the performance of the Cloud and
the other consists in evaluate the correctness of the received RFID events.

\subsection{Cloud Performance Evaluation}
\label{subs:cloud_performance_evaluation}
An aspect that is important concerns with the performance of the Cloud regarding the amount of data generated by the events. Cloud computing creates an illusion that
available computing resources on demand are infinite \cite{armbrust2009m}. However, with the increase of the amount of events, we must measuring these computing resources
in order to determine if the system is fulfilling the \textit{QoS} requirements. To perform the evaluation of the behaviour and performance of the Cloud we need to measure
some system metrics such:
\begin{itemize}
  \item \textit{CPU Utilization:} CPU Utilization indicates the percentage of time that the CPU was working at the instances in the Cloud. Normally, this metric
  is available through the Cloud providers. Usually, the range of this metrics is given in percentage that can vary between 0-100\%.
  \item \textit{Memory Usage:} Memory Usage indicates the amount of memory that is consumed by the system in a given period of time. The range of this metric is given in
  MBytes.
  \item \textit{System Load:} System Load is metric that indicates the general state of the system. This metric estimates the general performance of the system by measuring
  the number of received events. The range of this metric vary between 0 and 1. When this metric has a value of 0, it means that the system is not receiving any events at
  the time. If this metric has a value of 1, it means that the system is overloaded, consequently the CPU Utilization and Memory Usage metrics are close to the maximum value.
\end{itemize}

\subsection{RFID Evaluation}
\label{sub:rfid_evaluation}
RFID has its particular characteristics, which makes that traditional event processing systems cannot support them. Furthermore, RFID
events are temporal constrained \cite{wang2006bridging}. Temporal constraints as the time interval between two events and the time interval
for a single event are critical to event detection. In order to assure the correctness of these received events, some conditions must be defined
regarding the time interval that between the events. Temporal constraints are not the only cause that can compromise the correctness of received
the RFID events such as collisions on the air interface, tag detuning and tag misalignement \cite{floerkemeier2004issues}.\\

The metric used to evaluate the RFID performance is proposed by Correia \cite{Correia:Thesis:2014}. The \textit{PresenceRate} is a metric that takes in account the reported
time of reading a RFID tag. This metric measures the ratio between the time that a given object spent in a given area and the expected spent time. Correia
et al. \cite{Correia:Thesis:2014} defines that the expected behaviour of this metric is given by the following values:
\begin{itemize}
  \item \textit{PresenceRate = 0}: the object was not detected by the system.
  \item \textit{0 $<$ PresenceRate $<$ 1}: there are false negative readings and they were reported.
  \item \textit{PresenceRate = 1}: ideal scenario where the reported time is exactly the same as expected.
  \item \textit{PresenceRate $>$ 1}: there are false positive readings over-estimating the time spent by the object in the read area.
\end{itemize}

In order to perform a general evaluation of the system the we can establish a relation between the \textit{PresenceRate} and the metrics used to evaluate the Cloud
performance. The scenario where the \textit{PresenceRate} is larger then 1 can be used to determine if the system is overloaded. In this scenario the metrics of
\textit{PresenceRate} and \textit{SystemLoad} must be compared to verify this condition. If the \textit{SystemLoad} value is equal to 1, it means that the system is
operating in it maximum capacity and the amount of resources available is not enough to process all the requests. Otherwise this comparison can determine of the system
is overpowered. If the \textit{PresenceRate} is equal to 1 and the \textit{SystemLoad} is close to 0, it means that the available resources are more than the required to
process all the requests.
