\section{Solution Architecture}
\label{sec:solution_architecture}
A typical scenario for an IoT application consists of a smart space that contains several smart objects and all
the infrastructure required to support this application, namely RFID tags, sensors, readers and servers, as
illustrated in Figure \ref{fig:smart-space}. This scenario presents several issues regarding the deployment
of the application, the low scalability, the costs of infrastructure and the maintenance of the same.
% Typical Smart Space Scenario
\begin{figure}[h!]
  \centering
  \includegraphics[width=\textwidth]{./images/smart-space}
  \caption{Typical smart space scenario, using RFID technology.}
  \label{fig:smart-space}
\end{figure}\\
By converging the IoT applications with the Cloud Computing paradigm the objective is simplify this scenario by
leveraging the required infrastructure by these applications to the Cloud providers, as illustrated in
Figure \ref{fig:smart-space-cloud}, allowing to take advantage of the benefits offered by Cloud computing
(as referenced in \textbf{Section \ref{sub:cloud_computing}}).
% Cloud-based Smart Space Scenario
\begin{figure}
  \centering
  \includegraphics[width=\textwidth]{./images/smart-space-cloud}
  \caption{Cloud-based smart space.}
  \label{fig:smart-space-cloud}
\end{figure}\\
% -----------------------------------------------
% STATE OF ART
% -----------------------------------------------
\subsection{State of the Art}
\label{sub:state_of_art}
Cloud-based IoT applications are the state-of-the-art of this kind of solutions. By leveraging
the infrastructure to the Cloud providers, these applications have a high-availabity, can dynamically
scale while spending a fraction compared with the traditional solutions. However, as earlier
mentioned the deployment of such applications still is an issue, due its complexity and
required manual intervention. The deployment of Cloud-based IoT applications usually are performed
through IT automation tools, such as Chef\footnote{http://www.chef.io} and Puppet\footnote{http://www.puppetlabs.com}.
These automate the deployment of applications. However, the components of the application and the relation between
themselves must be specified manually, which requires considerable manual work and expertise by the person that
is performing the deployment. The deployment process of these solutions is not the only existing  issue.
Monitoring the application life-cycle is a task that requires a lot of effort and expertise by the system
administrators.\\

In order to solve this problem, the adopted approach relies on using Cloud Orchestrator tools perform the
deployment of these applications. As mentioned in \textbf{Subsection \ref{subs:cloud_orchestration}},
these tools allows the specification of the application components and their relations in a high-level perspective
and to execute the management tasks required by the application during its life-cycle. As a matter of fact,
in a low-level perspective cloud orchestration tools express the high-level perspective defined by the user
into scripts that latter are executed using IT automation tools such as Puppet and Chef.
% -----------------------------------------------
% CLOUD OF THINGS ARCHITECTURE
% -----------------------------------------------
\subsection{Cloud of Things Architecture}
\label{sub:cloud_of_things_architecture}
The main objective of this Cloud4Things proposal is  to decrease the complexity of deployment and management
of IoT applications. To achieve this objective, Cloud of Things must enable non-technical users - the business
managers - to perform the monitoring of Cloud-based IoT applications as well defining Service Level Agreements
in a high-level way. In order execute these tasks, users must be able to interact with the application that
is running at the cloud in order to observe their state and to apply some decisions based on the performance
of the application. Thus, Cloud4Things must provide a service, e.g a GUI, that allows the users to perform
such actions. In Figure \ref{fig:cloud_of_things_architecture} we present the proposed Cloud4Things architecture.
\vspace{1in}
% Cloud of Things Architecture
\begin{figure}[h!]
  \centering
  \includegraphics[width=.8\textwidth]{./images/cloud-of-things-architecture}
  \caption{Cloud of Things Architecture.}
  \label{fig:cloud_of_things_architecture}
\end{figure}\\
The presented architecture is based on the LoM2HiS Framework \cite{emeakaroha2010low} architecture. In this
architecture the \textit{Services} component and the \textit{Run-time Monitor} represents the application layer
where services are deployed using a Web Service container. The \textit{Run-time Monitor} is responsible to monitor
the services based on the negotiated and agreed SLAs. After the Cloud provider agrees on the SLA terms, the
agreed SLAs are stored in the repository for service provisioning and the following steps are executed:
\begin{enumerate}
  \item The Cloud provider creates rules for the framework mappings using Domain Specific
  Languages\footnote{Domain Specific Languages are small languages that normally are tailored to a specific
  problem domain.} (DSLs).
  \item The customer requests the provisioning of an agreed service.
  \item Once the request is received, the run-time monitor loads the service SLA from the agreed SLA repository.
  \item The resource metrics are measured by monitoring agents, these metrics are stored in a raw format that
  later are accessed by the host monitor.
  \item The host monitor extracts metric-value pairs from the raw metrics and them transmits them periodically to
  the run-time monitor.
  \item After receiving the low-level metrics, the run-time monitor uses predefined mapping rules to map the
  low-level metrics into an equivalent form of the agreed SLA and them the resulting map is stored in the
  mapped metrics repository.
\end{enumerate}

In this architecture, the \textit{Run-time monitor} uses the mapped values to monitor the status of the deployed services. Once it detects that a SLA is violated, the \textit{Run-time monitor} must alert the customer of the violated SLA.
At this point the customer is responsible to take the decisions in order to correct the state of the system.
