%!TEX root = ../dissertation.tex

\begin{otherlanguage}{portuguese}
\begin{abstract}
% Enable page numbering
\abstractPortuguesePageNumber

\textit{Smart places} s\~ao um ecosistema composto de sensores, actuadores e infra-estrutura computacional
que adiquirem dados do ambiente circundante e usam estes dados para melhorar a experi\^encia
das pessoas que interagem com este lugar. O \textit{smart place} possui uma aplica\c{c}\~ao para a Internet das
Coisas (IoT) que \'e capaz de transformar estes dados em informa\c{c}\~ao. Por exemplo, leitores RFID podem
detectar um objecto que est\'a a aproximar-se e uma porta autom\'atica \'e aberta ap\'os o evento ser
processado.\\

Geralmente, estas aplica\c{c}\~oes s\~ao \textit{latency-sensitive} visto que as ac\c{c}\~oes
tem de ser rapidamente executadas, o que faz com que a infra-estrutura necess\'aria para
aprovisionar a aplica\c{c}\~ao tem de ser local, o que muitas vezes \'e ineficiente e dispendioso.
A \textit{Utility Computing} na nuvem pode ajudar a resolver este problema. Entretanto, os
requisitos de lat\^encia devem ser cuidadosamente avaliados. A Computa\c{c}\~ao em Nevoeiro \'e
um conceito recente que aproxima a nuvem, fornecendo comunica\c{c}\~ao de baixa lat\^encia para
aplica\c{c}\~oes e servi\c{c}os.\\

Neste trabalho foram implementadas duas abordagens para o \textit{deployment} de aplica\c{c}\~oes IoT
basedas na nuvem e no nevoeiro. Nosso cen\'ario \'e um armaz\'em automatizado que utiliza a plataforma
Fosstrak - uma implementa\c{c}\~ao \textit{open-source} de \textit{software} para o processamento de
eventos RFID - para rastrear os objectos do armaz\'em. N\'os comparamos a performance da lat\^encia
dos eventos para ambas as abordagens e tamb\'em a performance do armazenamento de dados.\\

Os resultados obtidos mostraram que a abordagem baseada em nevoeiro \'e mais adequada para
aplica\c{c}\~oes \textit{latency-sensitive}, apresentando uma melhor performance quando comparada
com a abordagem em nuvem.\\

% Keywords
\palavrasChave{Internet das Coisas, Computa\c{c}\~ao em Nuvem, Computa\c{c}\~ao em Nevoeiro,
\textit{Deployment} de Aplica\c{c}\~oes, Plataforma Fosstrak}
\end{abstract}
\end{otherlanguage}
