%!TEX root = ../dissertation.tex

\chapter{Conclusion}
\label{chapter:conclusion}

\section{Outline}
\label{sec:outline}
The \acrfull{IoT} is a paradigm where ordinary objects are continuously connected to the virtual
world and interacting with the surrounding environment. In order to make this environment more efficient,
\gls{IoT} applications need to understand the context where they are inserted and also to adapt to this
environment. Thus, \gls{IoT} applications usually need to provisioning its infrastructure
in the physical place. This requirement makes that \gls{IoT} applications presents scalability
problems and also make them cost ineffective. The Utility Computing in the cloud helps to solve this
problems by leveraging part of the infrastructure of \gls{IoT} applications to the cloud. However,
\gls{IoT} applications are latency sensitive and the cloud must be able to meet this requirement.\\

Our solution is to determine if a cloud-based approach is suitable to deploy an \gls{IoT} application
for a smart warehouse that is based on the \gls{RFID} technology. To achieve our goals, we will
compare two approaches to determine which one is able to meet the \gls{IoT} applications requirements.
The fog-based approach, which brings the cloud to the edge of the network and a traditional cloud-based
approach, where the application is fully deployed in the cloud. In the implementation phase, we chose
the Fosstrak platform as the \gls{RFID} middleware for our application. The deployment of the
application middleware was performed in a different manner according the approaches. In the cloud-based
approach the application middleware is in the cloud, while in the fog-based approach the middleware
is distributed between the fog and the cloud.\\

To evaluate our solution, we focused in two points where \gls{IoT} applications are sensitive. First,
we evaluated the latency interaction for the proposed approaches. We also evaluated the data storage
performance of the Fosstrak middleware. Regarding the latency interaction performance, the obtained
results confirmed our initial hypothesis and shows that a fog-based approach presents a better
performance for the network latency when compared with the cloud-based approach. Regarding the data
storage performance, the obtained results shows that the Fosstrak middleware can handle the amount
of data that is typically generated in a smart warehouse, but it can present some performance issues
if several users concurrently overflow the data repository with events.

\section{Limitations and Future Work}
\label{sec:limitations_and_future_work}
Although we accomplish the initial goals for our work, our solution still presents some aspects that
need to be improved and other aspects that we are not able to concretize.\\

First, our solution proposes that the \gls{RFID} application in deployed following a fog-based approach.
This means that we need to have a cloud close to the edge of the network and this cloud must meet
the same requirements of a remote cloud such as high scalability, security and multi-tenancy.
Unfortunately, we are not able to accomplish a fog that meet these requirements and in the solution
evaluation our fog was built on top of a traditional Virtual Machine. Other aspect in the evaluation
process that can be improved regards with the \textit{ECspecs} that were used to evaluate our solution.
For the future work, we want to evaluate the latency performance for our solution with \textit{ECspecs}
that presents smaller periods.\\

Another limitation regards with our evaluation process where virtual \gls{RFID} reader was used
instead of a physical reader. This means that we are not able to reproduce the environment conditions
of a real smart warehouse such as interferences in the \gls{RFID} tags antennas, network bandwidth
variations, etc. Reproducing a real smart warehouse with physical \gls{RFID} readers and antennas
will improve our evaluation process in order to obtain more accurate results.\\

Finally, in our solution we used to Docker containers to provisioning the Fosstrak software stack.
In the evaluation of our solution we deploy the containers in a \gls{EC2} \gls{VM}, which overlays two
different mechanisms of virtualization. Although we still are able to take advantage of some benefits
from the containers such as the portability, other benefits such as the low I/O and disk space are
hidden by the \gls{VM} hypervisor. A future improvement that can be made is to deploy the containers
on top of the bare-metal in order to improve the overall performance of the solution.
