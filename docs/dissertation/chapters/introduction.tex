%!TEX root = ../dissertation.tex

\chapter{Introduction}
\label{chapter:introduction}
In recent years, computing is becoming more ubiquitous in the physical world. This notion where
computational elements are embedded seamlessly in ordinary objects that are connected through a
continuous network was introduced many years ago \cite{weiser1991computer}. The progress
towards ubiquitous computing has been slower than expected, technology advances such as the mobile
Internet contributes to achieve this vision in which computational devices are able to communicate
between themselves from any part of the world \cite{gubbi2013internet}. In this vision, an ubiquitous
system is composed of physical items that are continuously connected to the virtual world and can act as
remotely and physical access points to Internet Services \cite{mattern2010internet}.\\

However, there are some challenges that must be addressed in order to make these \gls{ubicomp} systems
truly ubiquitous \cite{caceres2012ubicomp}. An important concern regards about ubiquitous data: \textit{Where it is located?},
\textit{Who can access it?} and \textit{How much time this data should persist?}. Also, ubiquitous systems
are constantly interacting with the surrounding environment, thus these systems need to understand
the context in that they are inserted and also to adapt to the changes that occur in this environment.
Another import concern regards about the infrastructure burden of the ubiquitous systems. These
systems requires low-latency interaction with users and environments, which implies that at least part
of an \gls{ubicomp} application needs to be tightly bounded to the local infrastructure of the interacting
environment. This requirement for local infrastructure is a barrier in the adoption of ubiquitous
systems in a large-scale perspective.\\

Recently this ubiquitous world is close to becoming reality thanks to the Utility Computing in the
cloud and the \gls{IoT}. In one hand, the utility computing provides the illusion of infinite computing
resources available on demand to the public users \cite{armbrust2010view}, which helps to reduce the
infrastructure burden of the ubiquitous systems. In the other hand the \gls{IoT} aims to solve a key
problem in wider adoption of ubiquitous systems, the tight coupling with a particular embedded
infrastructure. With the \gls{IoT} a variety of \textit{objects} or \textit{things} - such as \gls{RFID},
tags, sensors, actuators, etc. - will be able to interact with each other and cooperate with the
surrounding \textit{things} to reach common goals \cite{atzori2010internet}.\\

This recent progress in the utility computing and the Internet of Things has been contributing to grow the
\gls{ubicomp} infrastructure. However, it is possible to identify new challenges for the construction
of ubiquitous systems that arises from the aggregation of the Internet of Things in the utility
computing.

% Challenges
\begin{itemize}
  \item Low-latency Interaction:
  \item Data Processing:
  \item Infrastructure Provisioning:
\end{itemize}

% Objectives
\section{Objectives}
\label{section:objectives}

% Viability of IoT's cloud-based applications.
% Approach to provisioning the IoT middleware

% Contribution
\section{Contributions}
\label{scction:contributions}

% Document Structure
\section{Document Structure}
\label{section:structure}
The remainder of this document is organized as follows:
% Document structure item list
\begin{itemize}
  \item \textbf{Chapter \ref{chapter:background} Background}
  \item \textbf{Chapter \ref{chapter:cloud4things} Cloud4Things}
  \item \textbf{Chapter \ref{chapter:methodology} Evaluation Methodology}
  \item \textbf{Chapter \ref{chapter:results} Evaluation Results}
  \item \textbf{Chapter \ref{chapter:conclusion} Conclusion}
\end{itemize}
