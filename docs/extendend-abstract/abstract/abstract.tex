%!TEX root = ../article.tex

% Abstract
\begin{abstract}
Smart places are an ecosystem composed of sensors - e.g. RFID - actuators - e.g. automatic doors - and
computing infrastructure - e.g. servers - that are able to acquire data about the surrounding
environment and use that data to improve the experience of the people interacting with the place.
The smart place runs an Internet of Things (IoT) application that is able to transform that data into
knowledge. Usually, IoT applications are latency-sensitive and requires that its infrastructure is
provisioned in the place, which represents an infrastructure burden. The Utility Computing in the cloud
can help to solve this problem, but it will be able to meet the latency requirements of these applications?
The present work implemented two deployment approaches for IoT applications based in the cloud:
a fog-based approach and a cloud-based approach. Our base scenario is an automated warehouse that
uses RFID technology - e.g. Fosstrak platform - to track the objects that are moving
inside the warehouse. We compared the performance of the latency interaction of both approaches.
The results show that a fog-based approach is more adequate for latency-sensitive IoT applications,
presenting a better overall performance when compared with the cloud approach.
\end{abstract}
