\documentclass{../llncs2e/llncs}
\usepackage{inputenc}
\usepackage{graphicx}
\usepackage{caption}
% page numbering
\pagestyle{plain}
% change the spacing between the caption and the table
\captionsetup[table]{skip=10pt}
% -------------------------------------------------------------------------------------------------
% CLOUD4THINGS: automatic provisioning of smart places infrastructure
% -------------------------------------------------------------------------------------------------
\title{Cloud4Things: automatic provisioning of smart places infrastructure}
% -------------------------------------------------------------------------------------------------
% AUTHORS
% -------------------------------------------------------------------------------------------------
 \author{\Large Marcus Vin\'icius Paulino Gomes}
% -------------------------------------------------------------------------------------------------
% INSTITUTION
% -------------------------------------------------------------------------------------------------∫
 \institute{\large T\'ecnico Lisboa, Universidade T\'ecnica de Lisboa\\
 \email{\{marcus.paulino.gomes\}@tecnico.ulisboa.pt}}
% -------------------------------------------------------------------------------------------------
% DOCUMENT
% -------------------------------------------------------------------------------------------------
\begin{document}
\maketitle
% -------------------------------------------------------------------------------------------------
% ABSTRACT
% -------------------------------------------------------------------------------------------------
\begin{abstract}
Smart places are an ecosystem composed by sensors and actuators that are able to acquire knowledge
about its environment and also to adapt itself in order to improve the experience of the peoples
that lives in this environment. In this paper we propose Cloud4Things, a solution that automates
the provisioning of smart places infrastructure in the Cloud by relying on configuration management
tools. In our solution we also want that an efficient and portable infrastructure software stack,
in that way Cloud4Things will use Docker containers to virtualize the infrastructure stack. To
evaluate our solution we are using Fosstrak, an open source RFID software platform that implements
most of the Electronic Product Code standards. We will perform a qualitative evaluation by comparing
the required software stack by our solution to provisioning the infrastructure for an application
based on Fosstrak with other approaches, such as full Virtual Machines and TOSCA.
\end{abstract}
% -------------------------------------------------------------------------------------------------
% INTRODUCTION
% -------------------------------------------------------------------------------------------------
\section{Introduction}
\label{sec:introduction}
In this section we will describe the problem that we want to solve with Cloud4Things, i.e, automating
the configuration and provisioning of smart places applications based on the RFID technology in the
Cloud, in our case the Fosstrak platform, that implements the GS1 EPC Network specifications.

Cloud4Things will perform the automatic provisioning aiming the required software stack by the smart
place infrastructure and the application performance.

We also will give some examples of smart places and describe the life-cycle of a smart place
application and point out what are the stages that will be the target of our work.
% -------------------------------------------------------------------------------------------------
% RELATED WORK
% -------------------------------------------------------------------------------------------------
\section{Related Work}
\label{sec:related_work}
In this section we will describe the most relevant work that is related with Cloud4Things, starting
by the work developed by Guinard and Floerkmeier (EPC Cloud), and later by presenting the other
solutions related with our work such as TOSCA, automatic provisioning with configuration management
tools + VMs and proprietary solutions such as Amazon VM import and Amazon Elastic Beanstalk.
% -------------------------------------------------------------------------------------------------
% SOLUTION ARCHITECTURE
% -------------------------------------------------------------------------------------------------
\section{Solution}
\label{sec:solution}
In this section we will describe the solution of work, i.e, Cloud4Things architecture:
% -------------------------------------------------------------------------------------------------
% DOCKER
% -------------------------------------------------------------------------------------------------
\subsection{Docker}
\label{sub:docker}
% -------------------------------------------------------------------------------------------------
% AUTOMATIC PROVISIONING
% -------------------------------------------------------------------------------------------------
\subsection{Automatic Provisioning}
\label{sub:Automatic Provisioning}
% -------------------------------------------------------------------------------------------------
% SMART PLACE AUTOMATIC CONFIGURATION
% -------------------------------------------------------------------------------------------------
\subsection{Smart Place Automatic Configuration}
\label{sub:Smart Place Automatic Configuration}
% -------------------------------------------------------------------------------------------------
% QUALITATIVE EVALUATION
% -------------------------------------------------------------------------------------------------
\section{Qualitative Evaluation}
\label{sec:qualitative_evaluation}
In this section we will perform a qualitative evaluation of the developed work until the date. The
evaluation consists in compare the efficiency of the software stack required by our solution with
the other alternative solutions that are mentioned in the Related Work.

Another component that can be evaluated is the process to configure a smart place based on the
Fosstrak platform. In the standard solution there are two implemented clients that can be used to
configure the readers and events that belongs to a smart place. However, configure a smart place
with those clients is a manual process that is ineffective and error prone. In order to turn this
configuration process more automatic, our approach is to provide a client application that allows
the user to describe the smart place configuration in a JSON file, that later will be used by the
client to setup the readers and events in the application that is running in the Cloud.
% -------------------------------------------------------------------------------------------------
% CONCLUSION
% -------------------------------------------------------------------------------------------------
\section{Conclusion}
\label{sec:conclusion}
In this section we will present the conclusion of the paper and also we will discuss about some
enhancements of our work.

At the top of the Future Work list is the measure of the elasticity of the application regarding
the events that occur in the physical world. Currently our solution is centralized, i.e, all the
container that runs the Fosstrak application are located in the same machine, a important component
of our work is to compare the current approach with a distributed solution. Finally, the client
application still is textual, in the future a important feature that can be added to our solution
is a GUI that allows the user to have a more visual perspective of the configuration of the smart
place. 
% -------------------------------------------------------------------------------------------------
% BIBLIOGRAPHY
% -------------------------------------------------------------------------------------------------
%\bibliographystyle{ieeetr}
%\bibliography{}
\end{document}
