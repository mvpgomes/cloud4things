% -------------------------------------------------------------------------------------------------
% INTRODUCTION
% -------------------------------------------------------------------------------------------------
\section{Introduction}
\label{sec:introduction}
In recent years, computing is becoming more ubiquitous in the physical world. This
notion where computational elements are embedded seamlessly in ordinary objects that
are connected through a continuous network was introduced many years ago\cite{weiser1991computer}.
Despite the progress towards ubiquitous computing has been slower than the expected,
technology advances such as the creation of the Internet contributes to achieve this vision in
which individual devices are able to communicate between themselves from any part of the
world\cite{gubbi2013internet}.

Recently, this ubiquitous world is close to becoming reality thanks to the Internet of Things
and Cloud Computing \cite{caceres2012ubicomp}. This world where things are connected
through a continuous network is a vision thats represents the Internet of Things (IoT).
In this vision, physical items are continuously connected to the virtual world and can
act as remotely physical access points to Internet Services. The Internet of Things would
make computing truly ubiquitous \cite{mattern2010internet}.

A common scenario where the Internet of Things paradigm is applied are smart places \cite{atzori2010internet}.
Smart places are an ecosystem composed of sensors - e.g. RFID tags - actuators - e.g. automatic doors -
and computing infrastructure - e.g. cloud servers - that are able to acquire data about the surrounding
environment and use that data to improve the experience of the people using the place \cite{cook2004smart}.
The data acquired by the sensors needs to be collected, interpreted and transformed into information that
is used to gather knowledge about the smart place. Fosstrak is a complete example of a platform that
can be used to transform the RFID data into information, since that implements most of the EPC Network
standards.

An example of a smart place is a smart warehouse. In the smart warehouse, products and objects
are identified with RFID tags. The data transmitted by the tags is processed to the computing
infrastructure of the smart place and then transformed into information. For instance,
Fosstrak can use this information to determine the objects that enter and leaves the smart place.

A common issue regarding RFID-based smart places concerns with the provisioning of the infrastructure of
a smart place. The current solution requires a physical infrastructure that is cost ineffective and presents
a low scalability. To solve those problems, an alternative is to allocate the smart place infrastructure
in the cloud. However, provisioning the smart place infrastructure in the cloud still is a manual
process that requires considerable effort and expertise to be executed.

In order to solve those problems we propose Cloud4Things, a solution that automates the provisioning of RFID
software in the cloud by relying on configuration management tools that leverage existing stacks.
With Cloud4Things we want to support the life-cycle of smart place applications, from the initial
provisioning to day-to-day operations, such as application management and QoS monitoring.

The remainder of this paper is organized as the follows. In Section \ref{sec:related_work} we
present the related work in the area. Section \ref{sec:solution} presents a description of our
solution architecture and the current implemented prototype. In Section \ref{sec:qualitative_evaluation}
we will perform a qualitative evaluation and compare our solution with the current state of the art.
Section \ref{sec:conclusion} presents the conclusion of this paper.
