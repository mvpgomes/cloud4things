% -------------------------------------------------------------------------------------------------
% INTRODUCTION
% -------------------------------------------------------------------------------------------------
\section{Introduction}
\label{sec:introduction}
In recent years computing is becoming more ubiquitous in the physical world. This
notion of ubicomp was introduced by Weiser \cite{weiser1999origins}, where the concept of
smart environment was defined as ``\textit{a physical world richly and invisibly
interwoven with sensors, actuators, displays, and computational elements, embedded
seamlessly in the everyday objects of our lives and connected through a continuous network}''.
Through the years, technology advances such as the creation of the Internet contributes
to achieving the ubicomp's vision which enables individual devices to communicate between
themselves from any part of the world \cite{gubbi2013internet}. In Weiser's vision of
ubiquitous computing, information seamlessly moves in and out of attention as automation
gives way to human interaction \cite{weiser1991computer}.

The progress towards ubiquitous computing has been slower than expected \cite{greenfield2010everyware},
another approaches for ubiquitous computing that contrasts with Weiser vision becomes relevant.
Rogers proposes an approach which focuses in designing technologies for engaging user
experiences that, in a creative and constructive way, extend the peoples' capabilities \cite{rogers2006moving}.
Rogers points that ubiquitous technologies can be developed not only for individuals,
but for particular domains that can be set up and customized by an organization according
its needs, such as agriculture, retailing and logistics.

The main difference between Weiser's and Rogers vision concerns with the user experience.
In Weiser point of view computers take the initiative to act on people's behalf \cite{tennenhouse2000proactive},
while in Rogers vision is not the computer that is proactive, the user assumes that role.

Actually, Weiser vision is close to becoming reality thanks to the Internet of Things
and Cloud Computing \cite{caceres2012ubicomp}. This world where things are connected
through a continuous network is a vision thats represents the Internet of Things (IoT).
In this vision, physical items are continuously connected to the virtual world and can
act as remotely physical access points to Internet Services. The Internet of Things would
make computing truly ubiquitous \cite{mattern2010internet}.

A common scenario where the Internet of Things paradigm is applied are smart places \cite{atzori2010internet}.
Smart places are an ecosystem composed of sensors - e.g. RFID tags - actuators - e.g. automatic doors -
and computing infrastructure - e.g. cloud servers - that are able to acquire data about the surrounding
environment and use that data to improve the experience of the people using the place \cite{cook2004smart}.
The data acquired by the sensors needs to be collected, interpreted and transformed into information that
is used to gather knowledge about the smart place. Fosstrak is a complete example of a platform that
can be used to transform the RFID data into information, since that implements most of the EPC Network
standards.

An example of a smart place is a smart warehouse. In the smart warehouse, products and objects
are identified with RFID tags. The data transmitted by the tags is processed to the computing
infrastructure of the smart place and then transformed into information. For instance,
Fosstrak can use this information to determine the objects that enter and leaves the smart place.


In this section we will describe the problem that we want to solve with Cloud4Things, i.e, automating
the configuration and provisioning of smart places applications based on the RFID technology in the
Cloud, in our case the Fosstrak platform, that implements the GS1 EPC Network specifications.

Cloud4Things will perform the automatic provisioning aiming the required software stack by the smart
place infrastructure and the application performance.

We also will give some examples of smart places and describe the life-cycle of a smart place
application and point out what are the stages that will be the target of our work.
