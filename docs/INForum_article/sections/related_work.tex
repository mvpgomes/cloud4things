% -------------------------------------------------------------------------------------------------
% RELATED WORK
% -------------------------------------------------------------------------------------------------
\section{Related Work}
\label{sec:related_work}


% ----------------------------------------
% INTERNET OF THINGS AND CLOUD COMPUTING
% ----------------------------------------
\subsubsection{EPC Cloud}
\label{subs:epc-cloud}
In RFID-based IoT applications, Guinard et al. \cite{guinard2011cloud} point out that the
deployment of RFID applications are cost-intensive mostly because they involve the
deployment of often rather large and heterogeneous distributed systems. As a consequence,
these systems are often only suitable for big corporations and large implementations and
do not fit the limited resources of small to mid-size businesses and small scale applications
both in terms of required skill-set and costs. To address this problem, Guinard et al. propose
a Cloud-based solution that integrates virtualization technologies and the architecture of
the Web and its services. The case of study consists in a IoT application that uses RFID technology
to substitute existing Electronic Article Surveillance (EAS) technology, such as those used in
clothing stores to track the products. In this scenario they applied the Utility Computing blueprint
to the software stack required by the application using the AWS platform and the EC2 service. To
evaluate the Cloud-based solution, two prototypes was successfully implemented to prove that
the pain points of the RFID applications can be relaxed by adopting the proposed solution.\\

% ---------------------------------------------
% TOSCA
% ---------------------------------------------
\subsubsection{TOSCA}
\label{subs:tosca}
TOSCA (Topology and Orchestration Specification for Cloud Applications) \cite{li2013towards} is
proposed in order to improve the reusability of service management processes and automate IoT application
ratified by OASIS in deployment in heterogeneous environments. TOSCA is a new cloud standard to formally
describe the internal topology of application components and the deployment process of Cloud applications.
The structure and management of IT services is specified by a meta-model, which consists of
a \textit{Topology Template}, that is responsible for describing the structure of a service, then there
are the \textit{Artifacts}, that describe the files, scripts and software components necessary to be
deployed in order to run the application, and finally the \textit{Plans}, that defined the management process
of creating, deploying and terminating a service. The correct topology and management procedure can be inferred
by a TOSCA environment just by interpreting the topology template, this is known as ``declarative" approach.
Plans realize an ``imperative" approach that explicitly specifies how each management process should be done.
The topology templates, plans and artifacts of an application are packaged in a Cloud Service Archive (.csar file)
and deployed in a TOSCA environment, which is able to interpret the models and perform the specified management
operations. These .csar files are portable across different Cloud providers, which is a great benefit in terms
of deployment flexibility. To evaluate its feasibility, TOSCA was used in the specification of a typical
application in building automation. an application to control an Air Handling Unit (AHU). The common IoT
components, such as gateways and drivers will be modeled, and the gateway-specific artifacts that are
necessary for application deployment will also be specified. By archiving the previous specifications
and corresponding artifacts into a .csar file, and deploying it in a TOSCA environment, the deployment
of AHU application onto various gateways can be automated. As a newly established standard to counter
growing complexity and isolation in cloud applications environments, TOSCA is gaining momentum in industrial
adoption as well for academic interest.\\

Talk about proprietary solutions as Amazon VM import and Amazon Elastic Beanstalk.
